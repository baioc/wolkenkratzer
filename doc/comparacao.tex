O paradigma lógico mostrou-se especialmente útil para resolver o problema deste trabalho, com diversas construções da linguagem atuando para facilitar o processo de resolução do \textit{puzzle}.
Além disso, a depuração do programa declarativo foi simples e intuitiva: tendo a noção de que cada regra torna o conjunto de soluções sempre mais específico, basta comentar temporariamente alguns trechos de código para observar os resultados parciais das regras que foram mantidas.

Em comparação com o paradigma funcional, foi destacado o poder e a generalidade dos predicados da programação lógica, afinal, funções representam apenas um subconjunto das relações no geral.
A facilidade obtida com programação de restrições em Prolog permitiu implementar o resolvedor para o quebra-cabeça de maneira mais dinâmica e em muito menos tempo do que em Haskell ou em Lisp.
Destaca-se entretanto, que a natureza do problema o torna especialmente compatível com o paradigma utilizado: outros programas eventualmente não compartilhariam de tantas vantagens como este o fez.
