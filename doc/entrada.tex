O resolvedor encontra-se abstraído no predicado \textit{wolkenkratzer/4}, portanto para utilizar o programa é necessário iniciar o interpretador de Prolog (SWI-Prolog, neste caso) e carregar as definições.
Em seguida, basta fornecer os parâmetros que definem uma instância do \textit{puzzle} e o sistema Prolog irá inferir as soluções existentes.
Segue a seguir uma consulta exemplificando o uso do programa para o quebra-cabeça ilustrado na Figura \ref{fig:example}.
Destaca-se que o programa também funciona para tabuleiros parcialmente resolvidos, bastando substituir valores conhecidos diretamente no tabuleiro.

\begin{minted}[style=mannie]{prolog}
%% consulta de entrada
?- Board =  [[_,_,_,_,_],
             [_,_,_,_,_],
             [_,_,_,_,_],
             [_,_,_,_,_],
             [_,_,_,_,_]],
   Board = [A,B,C,D,E],
   wolkenkratzer(
       Board,
       (
           [0,3,3,0,0], % borda superior
           [1,4,3,2,0], % borda esquerda
           [0,2,2,2,1], % borda inferior
           [3,2,3,0,1]  % borda direita
        ),
       5, false % altura maxima 5, sem diagonais
   ).

%% saida fornecida no interpretador
Board = [A,B,C,D,E],
A = [5, 3, 1, 4, 2],
B = [1, 2, 3, 5, 4],
C = [3, 4, 5, 2, 1],
D = [4, 5, 2, 1, 3],
E = [2, 1, 4, 3, 5] .
\end{minted}
